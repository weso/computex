\section{Introduction}

Publishing statistical data is a promising domain where linked data approaches
con offer a number of advantages. 

\TODO{a paragraph about linked data and RDF}

The SPARQL~\cite{SPARQL10} query language has been a successful technology to
increase the adoption RDF. 
The current SPARQL 1.1~\cite{SPARQL11} has added
new expressivity levels. 


\TODO{Talk about index data in general}

The Web Index project (\url{http://thewebindex.org}) of 2012
offered a data portal\footnoteUrl{http://data.webfoundation.org} whose data
was obtained by transforming the raw data and the computation values from Excel 
sheets to RDF\cite{Alvarez13}. 
In the new version of that data portal, we are planning to automate the
validation and even the computation of the index data from the raw data. 

\TODO{Talk a little bit about the web index project, link to our publication}

\TODO{Add an overview of our approach\ldots}

We are currently implementing the new version of the Web Index data portal using
this approach. At this moment, we have a running example and a validator
implemented in Scala. 
The source code and the example are available at
\url{https://github.com/weso/computex}.

Along the paper we will use Turtle and SPARQL notation and assume that the
namespaces have been declared using the most common prefixes found in
\url{http://prefix.cc}.
