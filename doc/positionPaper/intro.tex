\section{Introduction}

The creation and use of quantitative indexes is a widely 
accepted practice that has been applied to numerous domains like 
Bibliometrics (Impact factor), 
research and academic performance 
  (H-Index or Shanghai rankings), 
cloud computing (Global Cloud Index, by CISCO), 
etc.
We consider that those indexes and rankings could benefit from a 
 Linked Data approach where the rankings could be seen, tracked and 
 verified by their users.

We participated in the Web Index project
(\url{http://thewebindex.org}), which created an index to measure 
 the Web impact in different countries.
The 2012 version offered a data
portal\footnoteUrl{http://data.webfoundation.org} whose data was obtained 
by transforming the raw observations and precomputed values 
from Excel sheets to RDF~\cite{Alvarez13}. 
In the 2013 version of that data portal, we are working on 
both validating and computing observations to automatically derivate 
and populate new values to automate the validation and even the 
generation of the index from raw data.

We have defined a generic vocabulary 
of computational index structures which could be applied to compute and validate any other kind of
index and can be seen as an instance of the RDF Data Cube
vocabulary~\cite{Cube}.
The validation process employs SPARQL~\cite{SPARQL11} queries to model the 
 different integrity constraints and computation steps in a declarative way.

At this moment, we have a running example and a validator which 
 reads and executes the SPARQL queries. 
 Source code and some examples are available
 at~\url{https://github.com/weso/computex}. 
 Although our prototype validator has been implemented in Scala, 
 our approach is independent of any programming language 
 as far as it can load and execute SPARQL 1.1 queries.

 Along the paper we will use Turtle and SPARQL notation and assume that the
namespaces have been declared using the most common prefixes found in
\url{http://prefix.cc}.
