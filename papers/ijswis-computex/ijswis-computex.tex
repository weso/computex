\documentclass[conference]{IEEEtran}
% Add the compsoc option for Computer Society conferences.
%
% If IEEEtran.cls has not been installed into the LaTeX system files,
% manually specify the path to it like:
% \documentclass[conference]{../sty/IEEEtran}





% Some very useful LaTeX packages include:
% (uncomment the ones you want to load)


% *** MISC UTILITY PACKAGES ***
%
%\usepackage{ifpdf}
% Heiko Oberdiek's ifpdf.sty is very useful if you need conditional
% compilation based on whether the output is pdf or dvi.
% usage:
% \ifpdf
%   % pdf code
% \else
%   % dvi code
% \fi
% The latest version of ifpdf.sty can be obtained from:
% http://www.ctan.org/tex-archive/macros/latex/contrib/oberdiek/
% Also, note that IEEEtran.cls V1.7 and later provides a builtin
% \ifCLASSINFOpdf conditional that works the same way.
% When switching from latex to pdflatex and vice-versa, the compiler may
% have to be run twice to clear warning/error messages.






% *** CITATION PACKAGES ***
%
\usepackage{cite}
% cite.sty was written by Donald Arseneau
% V1.6 and later of IEEEtran pre-defines the format of the cite.sty package
% \cite{} output to follow that of IEEE. Loading the cite package will
% result in citation numbers being automatically sorted and properly
% "compressed/ranged". e.g., [1], [9], [2], [7], [5], [6] without using
% cite.sty will become [1], [2], [5]--[7], [9] using cite.sty. cite.sty's
% \cite will automatically add leading space, if needed. Use cite.sty's
% noadjust option (cite.sty V3.8 and later) if you want to turn this off.
% cite.sty is already installed on most LaTeX systems. Be sure and use
% version 4.0 (2003-05-27) and later if using hyperref.sty. cite.sty does
% not currently provide for hyperlinked citations.
% The latest version can be obtained at:
% http://www.ctan.org/tex-archive/macros/latex/contrib/cite/
% The documentation is contained in the cite.sty file itself.






% *** GRAPHICS RELATED PACKAGES ***
%
\ifCLASSINFOpdf
  % \usepackage[pdftex]{graphicx}
  % declare the path(s) where your graphic files are
  % \graphicspath{{../pdf/}{../jpeg/}}
  % and their extensions so you won't have to specify these with
  % every instance of \includegraphics
  % \DeclareGraphicsExtensions{.pdf,.jpeg,.png}
\else
  % or other class option (dvipsone, dvipdf, if not using dvips). graphicx
  % will default to the driver specified in the system graphics.cfg if no
  % driver is specified.
  % \usepackage[dvips]{graphicx}
  % declare the path(s) where your graphic files are
  % \graphicspath{{../eps/}}
  % and their extensions so you won't have to specify these with
  % every instance of \includegraphics
  % \DeclareGraphicsExtensions{.eps}
\fi
% graphicx was written by David Carlisle and Sebastian Rahtz. It is
% required if you want graphics, photos, etc. graphicx.sty is already
% installed on most LaTeX systems. The latest version and documentation can
% be obtained at: 
% http://www.ctan.org/tex-archive/macros/latex/required/graphics/
% Another good source of documentation is "Using Imported Graphics in
% LaTeX2e" by Keith Reckdahl which can be found as epslatex.ps or
% epslatex.pdf at: http://www.ctan.org/tex-archive/info/
%
% latex, and pdflatex in dvi mode, support graphics in encapsulated
% postscript (.eps) format. pdflatex in pdf mode supports graphics
% in .pdf, .jpeg, .png and .mps (metapost) formats. Users should ensure
% that all non-photo figures use a vector format (.eps, .pdf, .mps) and
% not a bitmapped formats (.jpeg, .png). IEEE frowns on bitmapped formats
% which can result in "jaggedy"/blurry rendering of lines and letters as
% well as large increases in file sizes.
%
% You can find documentation about the pdfTeX application at:
% http://www.tug.org/applications/pdftex





% *** MATH PACKAGES ***
%
%\usepackage[cmex10]{amsmath}
% A popular package from the American Mathematical Society that provides
% many useful and powerful commands for dealing with mathematics. If using
% it, be sure to load this package with the cmex10 option to ensure that
% only type 1 fonts will utilized at all point sizes. Without this option,
% it is possible that some math symbols, particularly those within
% footnotes, will be rendered in bitmap form which will result in a
% document that can not be IEEE Xplore compliant!
%
% Also, note that the amsmath package sets \interdisplaylinepenalty to 10000
% thus preventing page breaks from occurring within multiline equations. Use:
%\interdisplaylinepenalty=2500
% after loading amsmath to restore such page breaks as IEEEtran.cls normally
% does. amsmath.sty is already installed on most LaTeX systems. The latest
% version and documentation can be obtained at:
% http://www.ctan.org/tex-archive/macros/latex/required/amslatex/math/





% *** SPECIALIZED LIST PACKAGES ***
%
\usepackage{algorithmic}
% algorithmic.sty was written by Peter Williams and Rogerio Brito.
% This package provides an algorithmic environment fo describing algorithms.
% You can use the algorithmic environment in-text or within a figure
% environment to provide for a floating algorithm. Do NOT use the algorithm
% floating environment provided by algorithm.sty (by the same authors) or
% algorithm2e.sty (by Christophe Fiorio) as IEEE does not use dedicated
% algorithm float types and packages that provide these will not provide
% correct IEEE style captions. The latest version and documentation of
% algorithmic.sty can be obtained at:
% http://www.ctan.org/tex-archive/macros/latex/contrib/algorithms/
% There is also a support site at:
% http://algorithms.berlios.de/index.html
% Also of interest may be the (relatively newer and more customizable)
% algorithmicx.sty package by Szasz Janos:
% http://www.ctan.org/tex-archive/macros/latex/contrib/algorithmicx/




% *** ALIGNMENT PACKAGES ***
%
\usepackage{array}
% Frank Mittelbach's and David Carlisle's array.sty patches and improves
% the standard LaTeX2e array and tabular environments to provide better
% appearance and additional user controls. As the default LaTeX2e table
% generation code is lacking to the point of almost being broken with
% respect to the quality of the end results, all users are strongly
% advised to use an enhanced (at the very least that provided by array.sty)
% set of table tools. array.sty is already installed on most systems. The
% latest version and documentation can be obtained at:
% http://www.ctan.org/tex-archive/macros/latex/required/tools/


%\usepackage{mdwmath}
%\usepackage{mdwtab}
% Also highly recommended is Mark Wooding's extremely powerful MDW tools,
% especially mdwmath.sty and mdwtab.sty which are used to format equations
% and tables, respectively. The MDWtools set is already installed on most
% LaTeX systems. The lastest version and documentation is available at:
% http://www.ctan.org/tex-archive/macros/latex/contrib/mdwtools/


% IEEEtran contains the IEEEeqnarray family of commands that can be used to
% generate multiline equations as well as matrices, tables, etc., of high
% quality.


\usepackage{eqparbox}
% Also of notable interest is Scott Pakin's eqparbox package for creating
% (automatically sized) equal width boxes - aka "natural width parboxes".
% Available at:
% http://www.ctan.org/tex-archive/macros/latex/contrib/eqparbox/





% *** SUBFIGURE PACKAGES ***
%\usepackage[tight,footnotesize]{subfigure}
% subfigure.sty was written by Steven Douglas Cochran. This package makes it
% easy to put subfigures in your figures. e.g., "Figure 1a and 1b". For IEEE
% work, it is a good idea to load it with the tight package option to reduce
% the amount of white space around the subfigures. subfigure.sty is already
% installed on most LaTeX systems. The latest version and documentation can
% be obtained at:
% http://www.ctan.org/tex-archive/obsolete/macros/latex/contrib/subfigure/
% subfigure.sty has been superceeded by subfig.sty.



%\usepackage[caption=false]{caption}
%\usepackage[font=footnotesize]{subfig}
% subfig.sty, also written by Steven Douglas Cochran, is the modern
% replacement for subfigure.sty. However, subfig.sty requires and
% automatically loads Axel Sommerfeldt's caption.sty which will override
% IEEEtran.cls handling of captions and this will result in nonIEEE style
% figure/table captions. To prevent this problem, be sure and preload
% caption.sty with its "caption=false" package option. This is will preserve
% IEEEtran.cls handing of captions. Version 1.3 (2005/06/28) and later 
% (recommended due to many improvements over 1.2) of subfig.sty supports
% the caption=false option directly:
%\usepackage[caption=false,font=footnotesize]{subfig}
%
% The latest version and documentation can be obtained at:
% http://www.ctan.org/tex-archive/macros/latex/contrib/subfig/
% The latest version and documentation of caption.sty can be obtained at:
% http://www.ctan.org/tex-archive/macros/latex/contrib/caption/




% *** FLOAT PACKAGES ***
%
\usepackage{fixltx2e}
% fixltx2e, the successor to the earlier fix2col.sty, was written by
% Frank Mittelbach and David Carlisle. This package corrects a few problems
% in the LaTeX2e kernel, the most notable of which is that in current
% LaTeX2e releases, the ordering of single and double column floats is not
% guaranteed to be preserved. Thus, an unpatched LaTeX2e can allow a
% single column figure to be placed prior to an earlier double column
% figure. The latest version and documentation can be found at:
% http://www.ctan.org/tex-archive/macros/latex/base/



\usepackage{stfloats}
% stfloats.sty was written by Sigitas Tolusis. This package gives LaTeX2e
% the ability to do double column floats at the bottom of the page as well
% as the top. (e.g., "\begin{figure*}[!b]" is not normally possible in
% LaTeX2e). It also provides a command:
%\fnbelowfloat
% to enable the placement of footnotes below bottom floats (the standard
% LaTeX2e kernel puts them above bottom floats). This is an invasive package
% which rewrites many portions of the LaTeX2e float routines. It may not work
% with other packages that modify the LaTeX2e float routines. The latest
% version and documentation can be obtained at:
% http://www.ctan.org/tex-archive/macros/latex/contrib/sttools/
% Documentation is contained in the stfloats.sty comments as well as in the
% presfull.pdf file. Do not use the stfloats baselinefloat ability as IEEE
% does not allow \baselineskip to stretch. Authors submitting work to the
% IEEE should note that IEEE rarely uses double column equations and
% that authors should try to avoid such use. Do not be tempted to use the
% cuted.sty or midfloat.sty packages (also by Sigitas Tolusis) as IEEE does
% not format its papers in such ways.





% *** PDF, URL AND HYPERLINK PACKAGES ***
%
\usepackage{url}
% url.sty was written by Donald Arseneau. It provides better support for
% handling and breaking URLs. url.sty is already installed on most LaTeX
% systems. The latest version can be obtained at:
% http://www.ctan.org/tex-archive/macros/latex/contrib/misc/
% Read the url.sty source comments for usage information. Basically,
% \url{my_url_here}.





% *** Do not adjust lengths that control margins, column widths, etc. ***
% *** Do not use packages that alter fonts (such as pslatex).         ***
% There should be no need to do such things with IEEEtran.cls V1.6 and later.
% (Unless specifically asked to do so by the journal or conference you plan
% to submit to, of course. )

%

\usepackage[figuresright]{rotating}
%\usepackage{makeidx}  % allows for indexgeneration
\usepackage{graphicx}
\usepackage[T1]{fontenc}
\usepackage[english]{babel}
\usepackage[utf8]{inputenc}

%\usepackage{natbib}
\usepackage{url}
\usepackage{rotating}

\usepackage{latexsym}
 \usepackage{amsmath}
% \usepackage{amssymb}
% \usepackage{amsthm}
%\usepackage{eurosans}

\usepackage{eurosym}

\usepackage{longtable}

\usepackage{listings}

\usepackage{color}
\usepackage{textcomp}


\definecolor{gray}{gray}{0.5}
\definecolor{green}{rgb}{0,0.5,0}

\usepackage{rotating}


% correct bad hyphenation here
\hyphenation{op-tical net-works semi-conduc-tor}


\begin{document}
%
% paper title
% can use linebreaks \\ within to get better formatting as desired
\title{Towards the structural validation and computation of statistical index data represented in RDF using SPARQL queries: The Computex approach}


% author names and affiliations
% use a multiple column layout for up to three different
% affiliations
\author{\IEEEauthorblockN{José Emilio Labra-Gayo}
\IEEEauthorblockA{WESO Research Group\\Department of Computer Science\\
University of Oviedo\\
Oviedo, Asturias, Spain, 33007\\
Email: labra@uniovi.es}
\and
\IEEEauthorblockN{Jose María Alvarez-Rodríguez}
\IEEEauthorblockA{Knowledge Reuse Group\\Department of Computer Science\\
Carlos III University of Madrid\\
Leganés, Madrid, Spain, 28911\\
Email: josemaria.alvarez@uc3m.es}
\and
\IEEEauthorblockN{César Luis Alvargonzález}
\IEEEauthorblockA{WESO Research Group\\Department of Computer Science\\
University of Oviedo\\
Oviedo, Asturias, Spain, 33007\\
Email: cesar.luis@weso.es}}

% conference papers do not typically use \thanks and this command
% is locked out in conference mode. If really needed, such as for
% the acknowledgment of grants, issue a \IEEEoverridecommandlockouts
% after \documentclass

% for over three affiliations, or if they all won't fit within the width
% of the page, use this alternative format:
% 
%\author{\IEEEauthorblockN{Michael Shell\IEEEauthorrefmark{1},
%Homer Simpson\IEEEauthorrefmark{2},
%James Kirk\IEEEauthorrefmark{3}, 
%Montgomery Scott\IEEEauthorrefmark{3} and
%Eldon Tyrell\IEEEauthorrefmark{4}}
%\IEEEauthorblockA{\IEEEauthorrefmark{1}School of Electrical and Computer Engineering\\
%Georgia Institute of Technology,
%Atlanta, Georgia 30332--0250\\ Email: see http://www.michaelshell.org/contact.html}
%\IEEEauthorblockA{\IEEEauthorrefmark{2}Twentieth Century Fox, Springfield, USA\\
%Email: homer@thesimpsons.com}
%\IEEEauthorblockA{\IEEEauthorrefmark{3}Starfleet Academy, San Francisco, California 96678-2391\\
%Telephone: (800) 555--1212, Fax: (888) 555--1212}
%\IEEEauthorblockA{\IEEEauthorrefmark{4}Tyrell Inc., 123 Replicant Street, Los Angeles, California 90210--4321}}




% use for special paper notices
%\IEEEspecialpapernotice{(Invited Paper)}




% make the title area
\maketitle


\begin{abstract}
%\boldmath
The creation and use of quantitative indexes is a widely accepted practice that
has been applied to numerous domains like Bibliometrics, research and academic, cloud
computing, etc. with the objective of compiling several key performance
indicators in just one value. In this context, policymakers are continuously
gathering and analyzing statistical  data with the aim of providing objective
measures about a specific policy, activity, product or service and making some
kind of decision. Nevertheless existing tools and techniques based on
traditional processes are preventing a  proper use of the new dynamic and data
environment avoiding more timely, adaptable and flexible (on-demand)
quantitative index creation. On the other hand, semantic-based technologies
emerge to provide the adequate building blocks  to represent domain-knowledge
and process data in a flexible fashion  using a common and shared data model. In
this context, the Computex vocabulary, designed on the top of the RDF Data Cube
Vocabulary, is introduced to model the structure and the computation process of
quantitative indexes. Furthermore a validation method and a service based on
establishing different RDF profiles through the automatic generation of SPARQL
queries is presented to asses the index structure, populate new values and
ensure data quality. The Computex approach is also applied to the Webindex
project with the aim of evaluation both: 1) the structure and the computation
process of this index and 2) the capabilities of SPARQL as a validation method
for the Web of Data. The proposed approach can be generalized to other domains
by defining new RDF profiles. In fact the Computex validation tool can be
employed as a generic RDF validation tool where the user can define its own
profiles for validation. Finally, some discussion, conclusions and future work
in the context of this work and data quality are also outlined.


\end{abstract}
% IEEEtran.cls defaults to using nonbold math in the Abstract.
% This preserves the distinction between vectors and scalars. However,
% if the conference you are submitting to favors bold math in the abstract,
% then you can use LaTeX's standard command \boldmath at the very start
% of the abstract to achieve this. Many IEEE journals/conferences frown on
% math in the abstract anyway.

% no keywords




% For peer review papers, you can put extra information on the cover
% page as needed:
% \ifCLASSOPTIONpeerreview
% \begin{center} \bfseries EDICS Category: 3-BBND \end{center}
% \fi
%
% For peerreview papers, this IEEEtran command inserts a page break and
% creates the second title. It will be ignored for other modes.
\IEEEpeerreviewmaketitle


\section{Introduction}
\section{Introduction}

The creation and use of quantitative indexes is a widely 
accepted practice that has been applied to numerous domains like 
Bibliometrics (Impact factor), 
research and academic performance 
  (H-Index or Shanghai rankings), 
cloud computing (Global Cloud Index, by CISCO), 
etc.
We consider that those indexes and rankings could benefit from a 
 Linked Data approach where the rankings could be seen, tracked and 
 verified by their users.

We participated in the Web Index project
(\url{http://thewebindex.org}), which created an index to measure 
 the Web impact in different countries.
The 2012 version offered a data
portal\footnoteUrl{http://data.webfoundation.org} whose data was obtained 
by transforming the raw observations and precomputed values 
from Excel sheets to RDF~\cite{Alvarez13}. 
In the 2013 version of that data portal, we are working on 
both validating and computing observations to automatically derivate 
and populate new values to automate the validation and even the 
generation of the index from raw data.

We have defined a generic vocabulary 
of computational index structures which could be applied to compute and validate any other kind of
index and can be seen as an instance of the RDF Data Cube
vocabulary~\cite{Cube}.
The validation process employs SPARQL~\cite{SPARQL11} queries to model the 
 different integrity constraints and computation steps in a declarative way.

At this moment, we have a running example and a validator which 
 reads and executes the SPARQL queries. 
 Source code and some examples are available
 at~\url{https://github.com/weso/computex}. 
 Although our prototype validator has been implemented in Scala, 
 our approach is independent of any programming language 
 as far as it can load and execute SPARQL 1.1 queries.

 Along the paper we will use Turtle and SPARQL notation and assume that the
namespaces have been declared using the most common prefixes found in
\url{http://prefix.cc}.


\section{Related Work}
\input{sections/related-work}

\section{The Computex Vocabulary}
\section{Example data and Index computation process}

Our data model consists of a list of observations which can be raw observations
obtained from an external source or computed observations derived from other
observations. An example observation can be:

\begin{lstlisting}[style=SPARQL]
obs:obsM23 a qb:Observation ;
 cex:computation [ a cex:Z-Score ; 
      cex:observation obs:obsA23 ; cex:slice slice:sliceA09 ; ] ;
 cex:value 0.56 ;
 cex:md5-checksum "2917835203..." ;
 cex:indicator indicator:A ;
 cex:concept country:ESP ;
 qb:dataSet dataset:A-Normalized ;
 # ... other declarations omitted for brevity
\end{lstlisting}

Where we declare that \lstinline|obs:obsM23| is an observation
 whose value is \lstinline|0.56| that has been obtained as the Z-Score
 of the observation \lstinline|obs:A23| using the slice
 \lstinline|slice:sliceA09|. The observations refers to indicator
 \lstinline|indicator:A|, to the concept \lstinline|country:ESP| and to the
 dataset \lstinline|dataset:A-Normalized|.
 
For each observation, we also add a value for 
\lstinline|cex:md5-checksum| which is obtained as a combination of the 
different values of the observation and allows a user to verify the
values asserted to that observation.


\section{Computex vocabulary}

The \emph{Computex} vocabulary is available
at~\url{http://purl.org/weso/computex}. It defines terms related to the
computation of statistical index data and is compatible with RDF Data Cube
vocabulary. Some terms defined in the vocabulary are:

\begin{itemize}
\item\textbf{\lstinline|cex:Concept|} represents the entities that we are
indexing.
In the case of the Web Index project, the concepts are the different countries.
In other applications it could be Universities, journals, services, etc.

\item\textbf{\lstinline|cex:Indicator|}. A dimension whose values add
information to the Index.
Indicators can be simple dimensions, for example: the mobile phone
suscriptions per 100 population, or can be composed from other
indicators. 

\item\textbf{\lstinline|qb:Observation|}. This is the same term as in the 
RDF Data Cube vocabulary. It contains values for the
properties: \lstinline|cex:value|, \lstinline|cex:indicator| 
and \lstinline|cex:concept|, etc. 
The value of a \lstinline|qb:Observation| can be a Raw value
   obtained from an external source or a computed value obtained from other
   observations.

\item\textbf{\lstinline|cex:Computation|}. We have declared the main computation
types that we needed for the WebIndex project, which have been summarized in
Table~\ref{table:computations}. That list of computation types is non-exhaustive
and can be further extended in the future. 

\item\textbf{\lstinline|cex:WeightSchema|} a weight schema for a list of
indicators. It consists of a weight associated for each indicator which can be
used to compute an aggregated observation.

\end{itemize}

\begin{table*}[t]
\label{table:computations}
\begin{center}
\begin{tabular}{ p{0.2\textwidth} p{0.5\textwidth} p{0.3\textwidth}}
\toprule
Computation & Description & Properties \\
\hline
Raw			& No computation. Raw value obtained from external source.
			&  \\
Mean	    & Mean of a set of observations 
			& \lstinline|cex:observation| \newline 
			  \lstinline|cex:slice| \\
Increment	& Increment an observation by a given amount 
			& \lstinline|cex:observation| \newline 
			  \lstinline|cex:amount|  \\
Copy		& A copy of another observation 
			& \lstinline|cex:observation| \\
Z-score		& A normalization of an observation using the values from a Slice. 
			& \lstinline|cex:observation| \newline 
			  \lstinline|cex:slice| \\
Ranking		& Position in the ranking of a slice of observations. 
			& \lstinline|cex:observation| \newline 
			  \lstinline|cex:slice| \\
AverageGrowth & Expected average growth of N observations
			  & \lstinline|cex:observations| \\
WeightedMean & Weighted mean of an observation
			& \lstinline|cex:observation| \newline
			  \lstinline|cex:slice|       \newline
			  \lstinline|cex:weightSchema| \\
\bottomrule\\
\end{tabular}
\caption{Some types of statistical computations}
\end{center}
\end{table*}

\section{Validation approach}

The validation approach employed in the 2012 WebIndex project was based on
 resource templates similar to the OSLC resource
 shapes\footnoteUrl{http://www.w3.org/2012/12/rdf-val/SOTA} and
 the MD5 checksum field. 
 Apart from that, we did not verify that the precomputed values imported from
 the Excel sheets really match the value that could be obtained by 
 following the declared computation process.

The new validation approach proposed in the paper goes a step forward. 
The goal is not only to check that a resource contains
 a given set of fields and values, but also that those values really match
 the values that can be obtained by following the declared computations.
 
The proposed approach has been inspired by the integrity
constraint specification proposed by the RDF Data Cube vocabulary 
which employs a set of SPARQL
 \lstinline|ASK| queries to check the integrity of RDF Data Cube data. 
 Although \lstinline|ASK| queries provide a good means to check integrity, in
 practice their boolean nature does not offer too much help when a 
 dataset does not accomplish with the data model.

 We decided to use \lstinline|CONSTRUCT| queries which, in case of error, 
  contain an error message and a list of error parameters that can help to spot
  the problematic data.

 We transformed the \lstinline|ASK| queries defined in the RDF Data Cube
 specification to \lstinline|CONSTRUCT| queries. For example, the
 query to validate the RDF Data Cube integrity constraint 4 (IC-4) is:
 
\begin{lstlisting}[style=SPARQL]
CONSTRUCT {
 [ a cex:Error ; cex:errorParam [cex:name "dim"; cex:value ?dim ] ;
   cex:msg "Every Dimension must have a declared range" . ]
} WHERE { ?dim a qb:DimensionProperty .
  FILTER NOT EXISTS { ?dim rdfs:range [] }
}
\end{lstlisting}
 
In order to make our error messages compatible with EARL~\cite{EARL}, we have
 defined \lstinline|cex:Error| as a subclass of \lstinline|earl:TestResult| and 
 declared it to have the value \lstinline|earl:failed| for the property
 \lstinline|earl:outcome|.
 
We have also created our own set of SPARQL \lstinline|CONSTRUCT| queries to
validate the \emph{Computex} vocabulary terms, specially the computation of index data.
For example, the following query validates that every observation 
  has at most one value.
 
\begin{lstlisting}[style=SPARQL]
CONSTRUCT {
 [ a cex:Error ; cex:errorParam  # ... omitted 
    cex:msg "Observation has two different values" . ]
} WHERE { ?obs a qb:Observation . 
 ?obs cex:value ?value1 . ?obs cex:value ?value2 .
 FILTER ( ?value1 != ?value2  )
}
\end{lstlisting}

Using this approach, it is possible to define more expressive validations.
For example, we are able to validate that an observation has been obtained as
the mean of other observations. 

\begin{lstlisting}[style=SPARQL]
CONSTRUCT {
 [ a cex:Error ; cex:errorParam # ...omitted 
   cex:msg "Mean value does not match" ] . 
} WHERE { 
    ?obs a qb:Observation ;
         cex:computation ?comp ;
         cex:value ?val .
  ?comp a cex:Mean .
  { SELECT (AVG(?value) as ?mean) ?comp WHERE {
     ?comp cex:observation ?obs1 .
	 ?obs1 cex:value ?value ;
  } GROUP BY ?comp } 
 FILTER( abs(?mean - ?val) > 0.0001)
}
\end{lstlisting}

\section{Expressivity limits of SPARQL queries}

Validating statistical computations using SPARQL queries offered 
 a good exercise to check SPARQL expressivity. Although we were able 
 to express most of the computation types, some of them had to employ functions
 that are not part of SPARQL 1.1 or had to be defined in a limited way. 
 In this section we review some of the challenges that we found.

\begin{itemize} 

\item The Z-score of a value $x_i$ is defined as $\frac{x - \bar{x}}{\sigma}$
where $\bar{x}$ is the mean and $\sigma=\sqrt{\frac{\sum_{i=1}^{N}(\bar{x}-x_i)^2}{N -
1}}$ is the standard deviation. To validate that computation using SPARQL
queries, it is necessary to employ the \lstinline|sqrt| function. 
This function is not available in SPARQL 1.1 although some implementations 
 like Jena
 ARQ\footnoteUrl{http://jena.apache.org/documentation/query/library-function.html} 
 provide it.

\item In order to validate the ranking of an observation (in which position it
appears in a list of observations), we have found two approaches. One is to
check all the observations that are below the value of that observation. 
This approach requires checking the value of each observation against all the
other values. The other approach is to use a subquery that groups all the
observations ordered by their value using the \lstinline|GROUP_CONCAT|. 
However, SPARQL does not offer a function to calculate the position
of a substring in a string\footnote{This function is called \lstinline|strpos| in PHP or \lstinline|indexOf| in Java}, 
so we divided the length of the substring before the concept's 
name by the length of the concept's name. 
This approach is more efficient but only works when all the names have
the same length.

\item Given a list of values $x_1,x_2\ldots{}x_n$ the expected value
$x_{n+1}$ can be extrapolated using the forward average growth formula: 
$x_n\times{\frac{\frac{x_{n}}{x_{n-1}}+\ldots{}+\frac{x_{2}}{x_1}}{n-1}}$. 
Accessing RDF collections in SPARQL 1.1 requires property paths 
and offers limited expressivity. In this particular case 
the query can be expressed 
as\footnote{This query was suggested by Joshua Taylor.}:

\begin{lstlisting}[style=SPARQL]
CONSTRUCT {
  # ... omitted for brevity
} WHERE { 
  ?obs cex:computation [a cex:AverageGrowth; cex:observations ?ls] ;
  cex:value ?val .
  ?ls rdf:first [cex:value ?v1 ] .
  { SELECT ( SUM(?v_n / ?v_n1)/COUNT(*) as ?meanGrowth) WHERE {
      ?ls rdf:rest* [ rdf:first [ cex:value ?v_n ] ; 
                      rdf:rest  [ rdf:first [ cex:value ?v_n1 ]]] .
  }} 
 FILTER (abs(?meanGrowth * ?v1 - ?val) > 0.001) }
\end{lstlisting}

\end{itemize}


\section{Use Case: The Webindex}





\section{Research Study}
\subsection{Design}
\subsection{Results and Discussion}



\section{Conclusions and Future Work}
\section{Conclusions}

Using SPARQL queries to validate and compute index data seems a promising use
case for linked data applications. 
Although we have successfully employed this approach to validate most of the
statistical computations we needed for the WebIndex project, we have found some
limitations in current SPARQL 1.1 expressivity with regards to built-in
functions on maths, strings and RDF Collections.

We consider that our approach may be of interest to the RDF Validation 
 workshop in 2 ways: firstly as a practical approach to validate RDF data
 which poses some expressivity challenges, and secondly, as a 
 use case with real data than can act as a benchmark to compare different
 validation strategies.

Our future work is to automate the declarative computation of index data
 from the raw observations and to check the performance using
 the Web Index data. 
We are also studying the feasibility of this approach 
 for online calculation of index scores and rankings. 
% \TODO{Visualization of computed values?}


\section*{Acknowledgment}
The research leading to these results has received funding from the European Union’s Seventh Framework Programme (FP7-PEOPLE-2010-ITN) 
under grant agreement n° 264840 and developed in the context of the workpackage 4 and more specifically under the project 
``Quality Management in Service-based Systems and Cloud Applications''.



\bibliographystyle{IEEEtran}
% argument is your BibTeX string definitions and bibliography database(s)
\bibliography{references}
\end{document}


